\documentclass[bachelor]{stpthesis}
\usepackage[utf8]{inputenc}
\usepackage[british,swedish]{babel}

% För bibliografin
\usepackage[round]{natbib}
\bibliographystyle{sweplnat}

\usepackage[babel,swedish=guillemets*]{csquotes}
\let\q\enquote

% Definierar kommandona \url och \path.
\usepackage{url}

% För IPA-exemplet
\usepackage{tipa}

% För longtable-exemplet
\usepackage{longtable}

% Logiska kommandon för att skriva om LaTeX-klasser och LaTeX-paket.
%
% I denna text nämner jag LaTeX-klasser flera gånger. Sådana
% vill jag skriva i annan stil (med \textsf). Istället för att skriva
% så väljer jag att definiera ett kommando \class{...} som gör samma
% sak. Om jag ångrar mig angående hur en klass ska skrivas ut behöver
% jag bara ändra på ett ställe. Samma med \pack{...} för paket.
% Enkla nya kommandon ser ut precis så här. Siffran inom hakparenteser
% anger hur många argument det nya kommandot tar och dessa läggs sedan
% in med #1, #2, osv.

\newcommand{\class}[1]{\textsf{#1}}
\newcommand{\pack}[1]{\textsf{#1}}

% Jag skapar en egen omgivning för citerad kod (Quoted CODE)
% och låter den vara ekvivalent med boxedverbatim.

\newenvironment{qcode}{\boxedverbatim}{\endboxedverbatim}
\bvtopofpage{\begin{center}\normalfont (forts.)\end{center}}

% Eftersom jag behöver så mycket kort verbatim-text gör jag så
% det räcker att omge text med ||.
\MakeShortVerb{\|}

% Använd smalt mellanslag i förkortningen "t.ex.".
\newcommand{\tex}{t.\,ex.}

% En del ord som inte blev bra avstavade
\hyphenation{kan-ske titel-sida titel-sidan harvard-modell-en stp-thesis
  pdf-latex stan-dard-inställ-ning-en efter-som eko-arean sats-ytan
  stycke
  biblio-grafi-stil biblio-grafi-stilen biblio-grafi-stilar
  biblio-grafi-stil-ar-na
}

% Helst ska man inte avstava namn på svenska alls, men att
% avstava sammansatta efternamn är rätt OK ändå.
\hyphenation{ström-quist alm-qvist dahl-löf}

\begin{document}
\author{Per Starbäck}
\supervisors{Ettan Handledare, Uppsala Universitet\\
  	     Tvåan Handledare, NLP Enterprises AB}
\title{Att skriva uppsatser med stpthesis}
\subtitle{En användarhandledning}
% \credits{30}

\maketitle
\frontmatter*

\begin{abstract}
Sammandrag skrivs med omgivningen |abstract|.
\end{abstract}

\begin{otherlanguage}{british}
\begin{abstract}
It can be useful to have several abstracts in different languages,
like this.
\end{abstract}
\end{otherlanguage}

\clearpage
\tableofcontents*

\chapter{Förord}

Tack till alla som agerat försökskaniner till tidigare versioner av
detta, samt alla andra som kommit med kommentarer!

(Detta är ett kapitel i uppsatsens \emph{frontmatter},
och därför får det inget nummer.)

\mainmatter*
\chapter{Inledning}

Denna text beskriver \class{stpthesis} -- en LaTeX-klass
att skriva STP-examensarbeten med. Dessutom nämns en del
saker som inte är speciella för \class{stpthesis} men som kan vara
särskilt användbara för STP:are som skriver examensarbete i LaTeX.

Denna text är själv skriven med den klassen, trots att den inte är ett
STP-examensarbete, för att visa hur det kan se ut.

\Citet{dahllof:exarbete} ger en lista med rekommenderad litteratur
för examensarbeten
\citep{dahllof:akupp, saxena:mallar, stromquist:upp},
och den har också använts vid framställandet av \class{stpthesis}.

Dessutom har allmän litteratur om typografi använts
\citep{hellmark:th, bringhurst:elements, lansburgh}.

För det som inte är specifikt för just \class{stpthesis} hänvisas i
första hand till kapitlet om LaTeX i kurskompendiet till
Introduktion till datateknik för språkvetare \citep{idslatex}.

Länk till detta och till en del andra LaTeX-resurser finns på
\url{http://stp.lingfil.uu.se/datorer/tex/}.

Klassen \class{stpthesis} bygger inte på någon av standardklasserna i
LaTeX, utan på klassen \class{memoir}, som har många fördelar jämfört
jämfört med standardklasserna.

\chapter{Uppsatsens delar}
\section{Ange klass}

En LaTeX-fil börjar med en deklaration av vilken
\emph{dokumentklass} som just detta dokument använder.
Med denna klass blir den raden
\begin{qcode}
\documentclass[bachelor]{stpthesis}
\end{qcode}

\noindent eller med |master| i stället för |bachelor|, beroende på vad
det är för slags uppsats.

Man kan lägga till fler \emph{väljare} (\emph{options})
genom att lista dem alla inom hakparenteserna, \tex\ som

\begin{qcode}
\documentclass[master,times,12pt]{stpthesis}
\end{qcode}

I detta exempel används väljarna |master|, |times| och |12pt|. Senare
nämns några
sådana väljare som du kan vilja använda under särskilda omständigheter,
men normalt behöver du bara ange |bachelor| eller |master|.

\section{Dokumenthuvudet}

Själva textens början visas med |\begin{document}|.
Avsnittet före det kallas för dokumenthuvudet (\emph{preamble}).
Början av den ser ut så här i ett arbete som är skrivet på svenska:

\begin{qcode}
\usepackage[utf8]{inputenc}
\usepackage[swedish]{babel}
\end{qcode}

Normalt är det en bra idé att även ta med
|\usepackage[T1]{fontenc}|
i sina LaTeX-filer, men det lägger \class{stpthesis} till
av sig självt, så det behövs inte här.
(Det är för att välja fonter där \tex\ \emph{åäö} är \emph{egna tecken}
och inte behöver konstrueras med accenter.)

\begin{itemize}
\item inputenc: Detta talar om vilken teckenkodning din infil
	använder sig av. Utan detta fungerar inte tecken som åäö.
        Det skadar inte att ange det även om du (i en fil på engelska)
        inte använder några \q{konstiga} tecken.
        Om din fil är i Latin1 istället för UTF8 så skriv |latin1|
        istället.
\item babel: Om du skriver uppsatsen på engelska behöver du inte ange
        detta, men du kan ange |british| eller |american| uttryckligen
        för att få motsvarande konventioner för avstavningar och
        datumformat. (Läs mer om att ha olika språk i uppsatsen i
        kapitel~\ref{chap:lang}.)
\end{itemize}

Det finns en mängd andra paket som kan kan vilja
använda i sitt dokument och som man i så fall tar in här
med |\usepackage|. En del tips på sådana följer senare.

\section{Början av dokumentet}

\begin{qcode}
\begin{document}
\author{Ellen Kaka}
\supervisors{Anna Handledare, Uppsala universitet\\
	     Bertil Handledare, NLP Enterprises AB}
\title{Palindrom i svenskan}
\date{1 april 2012}
\maketitle
\end{qcode}

Kommandot |\maketitle| genererar en titel utifrån information från de
övriga kommandona, som därför måste komma först.

Om du bara har \emph{en} handledare, så använd |\supervisor| istället.

Om du utelämnar |\date| kommer dagens datum att användas istället,
vilket väl är lämpligt innan du är klar med uppsatsen.
% , så ser du därmed när en viss version är gjord.

Med kommandot |\subtitle| kan du dessutom lägga till en undertitel.
% och med |\credits| kan du tala om hur många hp uppsatsen är på.

% (Det finns ett kommando |\authoremail| också för att ange ens
% datoradress, eftersom den nämndes på titelsidan tidigare, men
% numer har det ingen effekt.)

Om din titel är för lång för att rymmas på en rad, och du inte är nöjd
med de radbrytningar som blir, så lägg gärna själv in radbrytningar
med |\\| i den. Radbrytning av titlar och rubriker bör allra helst
göras för hand, med hänsyn tagen till fraserna i texten.

Efter titeln kommer

\begin{qcode}
\frontmatter*
\begin{abstract}
  Här ska uppsatsens sammandrag vara. Den skrivs dock
  lämpligen när resten av uppsatsen är färdigskriven.
\end{abstract}
	
\clearpage
\tableofcontents*

\clearpage
\listoffigures

\clearpage
\listoftables
\end{qcode}

Här kommer |\maketitle| att konstruera en titelsida med de data du
angav tidigare.

Innehållsförteckningen är obligatorisk. Om du ska ha särskilda
förteckningar över figurer och tabeller använder du även de sista
raderna i exemplet. Asterisken i slutet av |\tableofcontents*| gör att
innehållsförteckningen själv inte hamnar i innehållsförteckningen.
(Det finns även motsvarande asteriskformer
|\listoffigures*| och |\listoftables*|.)


\section{Onumrerade och numrerade kapitel}

Efter detta kommer kapitlen i din uppsats:
\begin{qcode}
\chapter{Förord}
...
	
\mainmatter*
\chapter{Introduktion}
...
\end{qcode}

Lägg in |\mainmatter*| före första kapitlet i huvudinnehållet. Det som
följer därefter blir kapitel~1, och tidigare kapitel blir onumrerade,
vilket är normalt för ett förord.

Om man istället använder |\frontmatter| och |\mainmatter| utan
stjärnor så kommer dessutom sidnumreringen att börja med~1 först där
kapitel~1 börjar och tidigare sidor får romerska sidnummer.
Det är mer traditionellt, men kan vara opraktiskt när man
läser uppsatsen på dator, eftersom det då finns två sätt att numrera
sidorna och man kan lätt göra fel när man ger ett kommando för att
bara skriva ut vissa sidor från filen.

\section{Slutet av uppsatsen}

Eventuella bilagor i uppsatsen anges som vanliga kapitel
med |\chapter|,
men skriv |\appendix| före det första, som därmed blir bilaga~A.

Med |\backmatter| talar du om att det är slut på numrerade kapitel.
Om du har ytterligare kapitel efter det kommer de att bli onumrerade
bilagor.

I slutet av uppsatsen har du också referensförteckningen.
Den tas upp i kapitel~\ref{chap:refs} på s.~\pageref{chap:refs}.

Alltihop avslutas med
\begin{qcode}
\end{document}
\end{qcode}

\chapter{Uppställningar och tabeller}
 
Klassen \class{stpthesis} har en del utökningar och ändringar jämfört
med hur uppställningar fungerar i standardklasserna.

En |tabular| som denna:

\begin{quote}
  \begin{tabular}{lll} \toprule
    \textsf{svenska} & \textsf{engelska} & \textsf{esperanto} \\
    \midrule
    ett & one & unu     \\
    två & two & du      \\
    tre & three & tri   \\
    \bottomrule
  \end{tabular}
\end{quote}
kan man göra så här:

\begin{qcode}
\begin{tabular}{lll} \toprule
  \textsf{svenska} & \textsf{engelska} & \textsf{esperanto} \\
  \midrule
  ett & one & unu       \\
  två & two & du        \\
  tre & three & tri     \\
  \bottomrule
\end{tabular}
\end{qcode}

Det som är speciellt i detta exempel är kommandona
|\toprule|, |\midrule| och |\bottomrule| som
ger lite bättre utseende än den linje |\hline| som annars är den
normala i LaTeX.
Typografisk litteratur brukar förespråka en sådan enkel layout utan
lodräta streck och oftast med bara tre vågräta streck -- en |\toprule|
överst, en |\bottomrule| underst och en |\midrule| mellan rubrik och
resten.

% \begin{table}
%   \caption{En linjerad tabell av ett slag som inte rekommenderas.}
%   \centering
%   \begin{tabular}{|l|l|l|} \hline
%     \textsf{svenska} & \textsf{engelska} & \textsf{esperanto} \\ \hline
%     ett & one & unu	\\ \hline
%     två & two & du	\\ \hline
%     tre & three	& tri	\\ \hline
%   \end{tabular}
%   \label{tab:linjerad}
% \end{table}

I många LaTeX-sammanhang ser man istället uppställningar med mycket
mer linjer, inklusive lodrätta linjer.
% , som tabell~\ref{tab:linjerad}.
Typografer brukar avråda från detta.

Denna uppställning (|tabular|) är inte en \emph{tabell} (|table|). Med
tabell menas här något som har en beteckning som \q{tabell~4.2} och
som har en lite friare placering och normalt hamnar högst upp eller
längst ner på en sida i närheten.

AucTeX förstår att man vill ha en |tabular| inne i en
|table|, så när man ger ett kommando för att lägga in en |table| så
får man en mall för en |tabular| på köpet.

När man lägger in en |table| med |C-c C-e| får man först en fråga
\emph{(Optional) Float Position:} som man kan svara med bara retur på.
(Så kan man göra med allt som står som \emph{Optional}.)\footnote{%
  Det står mer om detta val i avsnitt~\ref{sec:placering} om placering
  av figurer och tabeller.}
Sen kommer frågor om \emph{Caption} och om man vill centrera
tabellen (här är vanligtvis~\emph{y} ett bra svar). På frågan om
\emph{label} ger man denna tabell en kort etikett som man sedan
använder för att hänvisa till den med |\ref|. För tabeller brukar den
börja med~|tab:| vilket AucTeX lägger till av sig självt.

När man kommer till själva uppställningen får man bland annat en fråga
om \emph{Format} vilket gäller vilka kolumner som används, \tex\
\q{lll} för uppställningen ovan. (Tre vänsterställda kolumner.)
Bokstäverna i formatet är oftast |l| för vänstställd, |c| för
centrerad och |h| för högerställd kolumn.

% Observera att man brukar sätta en rubrik (|\caption|) \emph{över}
% tabellens innehåll till skillnad från figurer där bildtexten hamnar
% \emph{under} själva figuren. Etiketter för tabeller börjar
% konventionellt med |tab:|.

% För mer komplicerade tabeller, \tex\ där vissa celler är ihopslagna i
% vissa rader eller kolumner, får du läsa i andra texter.
Eftersom omgivningen |tabular| i \class{memoir} är utökad jämfört med den
som är standard i LaTeX, så nöj dig inte med allmänna
LaTeX-introduktioner utan se även \class{memoir}-dokumentationen
om du har behov utöver vad som står här.

\section{Breda tabeller}\label{sec:bred}

Flera har velat ha tabeller som blir för breda för papperet, eller i
alla fall för satsytan.
En sak man kan göra då är att välja en mindre text i tabellen, \tex\
med |\small|.
En annan möjlighet är att låta tabellen sticka ut i marginalerna
med hjälp av |usemargins|.
Omgivningen |usemargins| är speciell för \class{stpthesis} och använder sig
av |adjustwidth| i \class{memoir}, med vilket du kan styra beteendet
mer om det skulle behövas.

Den breda tabellen~\ref{tab:bred} här nedan är skriven med en
omgivning |usemargins| omkring hela |tabular|en.


\begin{table}[hb]\small
  \caption{En bred tabell.}
  \label{tab:bred}
  \begin{usemargins}
    \begin{tabular}{cccccccccccccccc}
      \toprule
       1 & 2 & 3 & 4 & 5 & 6 & 7 & 8
         & 9 & 10 & 11 & 12 & 13 & 14 & 15 & 16  \\
       \midrule
       ett & två & tre & fyra & fem & sex
           & sju & åtta & nio & tio & elva & tolv
           & tretton & fjorton & femton & sexton \\ 
       unu & du & tri & kvar & kvin & ses
           & sep & ok & na\u{u} & dek & dek unu & dek du
           & dek tri & dek kvar & dek kvin & dek ses \\
      \bottomrule
    \end{tabular}
  \end{usemargins}
\end{table}

% En tredje möjlighet för verkligt breda tabeller är att lägga in
% tabellen åt sidan istället.

\section{Långa tabeller}

Det kan också hända att man har en tabell som är för lång för att
rymmas på en sida. Då får man lov att lägga den ungefär som en
|tabular| direkt på stället, men istället med |ctabular|
(\emph{continous tabular}) som kan fortsätta över till nästa sida. Här
kommer en sådan. (I just detta fall hade det gå att få rum med hela på
samma sida, så det är inte så stor poäng med det.)

% En \caption kan bara användas till de rörliga tabellerna och
% figurerna. Här definieras och används ett kommando för fristående
% tabellrubriker som beter sig som tabellrubriker (dvs. hamnar
% i table-numreringen).
\newfixedcaption{\freetabcaption}{table}
\freetabcaption{Sambandet mellan semantisk roll och
  satsledsfunktion vid några vanliga verb.
  (Taget från \emph{Svenska Akademiens grammatik}, 4, s.~65.)
  Här har tabellen dubblerats för att bli längre.}
\begin{ctabular}{lllll}\toprule
\textbf{Verb} & \textbf{Agens} & \textbf{Upplevare}
	& \textbf{Föremål} & \textbf{Plats} \\
\midrule
\emph{undervisa} & subjekt & objekt  &        & adverbial \\
\emph{tömma}     & subjekt & --      & --     & objekt \\
\emph{hämta}     & subjekt & --      & objekt & adverbial \\
\emph{sätta}     & subjekt & --      & objekt & adverbial \\
\emph{älska}     & --      & subjekt & objekt & \\
\emph{upptäcka}  & --      & subjekt & objekt & \\
\emph{hamna}     & --      & --      & subjekt& adverbial \\
\emph{passera}   & --      & --      & subjekt (inanimat) & objekt \\
% Och en gång till bara för att göra det längre
\emph{undervisa} & subjekt & objekt  &        & adverbial \\
\emph{tömma}     & subjekt & --      & --     & objekt \\
\emph{hämta}     & subjekt & --      & objekt & adverbial \\
\emph{sätta}     & subjekt & --      & objekt & adverbial \\
\emph{älska}     & --      & subjekt & objekt & \\
\emph{upptäcka}  & --      & subjekt & objekt & \\
\emph{hamna}     & --      & --      & subjekt& adverbial \\
\emph{passera}   & --      & --      & subjekt (inanimat) & objekt \\
\bottomrule
\end{ctabular}

I detta exempel har denna |ctabular| fått en rubrik precis som om det
vore en |table|. Om man vill ha det så måste man trixa lite, så koden
för detta börjar som i figur~\ref{fig:ctabular}.

\begin{figure}[b]
  \centering
\begin{qcode}
\newfixedcaption{\freetabcaption}{table}
\freetabcaption{Sambandet mellan semantisk roll och
  satsledsfunktion vid några vanliga verb.
  (Taget från \emph{Svenska Akademiens grammatik}, 4, s.~65.)
  Här har tabellen dubblerats för att bli längre.}
\begin{ctabular}{lllll}\toprule
\textbf{Verb} & \textbf{Agens} & \textbf{Upplevare}
	& \textbf{Föremål} & \textbf{Plats} \\
\midrule
\emph{undervisa} & subjekt & objekt  &        & adverbial \\
...  
\end{qcode}
  \caption{Början av koden för \texttt{ctabular}-exemplet.}
  \label{fig:ctabular}
\end{figure}

Vill man ha mer finesser finns det andra möjligheter men då måste
man ladda extra paket. Två möjligheter är \pack{longtable} och
\pack{xtab} som den intresserade kan titta närmare på.

\chapter{Figurer och bilder}\label{chap:fig}

Figurer fungerar som tabeller vad gäller placering,
men har en egen omgivning |figure| för
att få egna beteckningar och egen numrering.
För breda figurer kan använda sig |usemargins| liksom för tabeller
(avsnitt~\ref{sec:bred}), och om du har problem med var figurerna
hamnar, så se avsnitt~\ref{sec:placering}.

Själva innehållet i figuren kan man göra på flera sätt, men ofta är
det en bild som man hämtat från en annan källa.
För att kunna inkludera bilder med |\includegraphics|
måste man normalt först ha laddat ett grafikpaket. Klassen
\class{stpthesis} har dock redan laddat paketet \pack{graphicx},
så detta behöver du inte göra.

Bilder som ska inkluderas bör vara PDF eller PNG.
För traditionell LaTeX (som skapar en |.dvi|-fil) är det
istället bäst att använda EPS (Encapsulated Postscript) för bilderna.
Programmet |convert| omvandlar mellan olika bildformat, \tex\
\begin{qcode}
convert foo.eps foo.png
\end{qcode}

(För att konvertera från EPS till PDF finns även
programmet \texttt{epstopdf}.)

I sin enklaste form inkluderar man en bild med något i stil med
\begin{qcode}
\includegraphics{gnu}
\end{qcode}
för en fil som \tex\ kan heta |gnu.png|.
(Det är bäst är på detta sätt inte ange filextensionen i argumentet
till |\includegraphics|. Då kommer den själv att välja den fil som
passar bäst ifall det finns flera.)

Det är vanligt är att man behöver skala figuren. Då kan man \tex\
skriva |\includegraphics[width=10cm]{gnu}| för att tala om att figurens
bredd ska vara 10\,cm.
Ibland kan det för en stor figur vara praktiskt att ange
|width=\linewidth| som ger samma bredd som texten har, eller
\tex\ |width=0.9\linewidth| för att få 90\,\% av detta.

För fler möjligheter om hur man inkluderar bilder, se
kapitel~11 i dokumentationen till \class{memoir}, samt
\q{Using Imported Graphics in LaTeX and pdfLaTeX}.

Den senare finns på nätet på
\url{http://www.ctan.org/tex-archive/info/epslatex.pdf}
samt utskriven i tidskriftsamlaren \q{\TeX} i stora datorsalen.
% [gammal version:]
% som finns i \path{/usr/share/texmf/doc/latex/graphics/epslatex.ps}
% på GNU/Linux-datorerna.
% Men versionen på nätet är nyare, så hänvisa till den istället.

\chapter{Lingvistik}

I språkvetenskap är det vanligt att man har numrerade språkexempel.
Om du har sådana i din uppsats kan du skriva dem med

\begin{qcode}
\begin{examples}
  \item Detta är en mening.
  \item Här är en annan mening. \label{ex:foo}
\end{examples}

Se särskilt exempel~\ref{ex:foo} ovan!
\end{qcode}

vilket resulterar i följande:

\begin{quote}
  \begin{examples}
  \item Detta är en mening.
  \item Här är en annan mening. \label{ex:foo}
  \end{examples}

  Se särskilt exempel~\ref{ex:foo} ovan!
\end{quote}

Ett ensamt exempel kan man alternativt skriva som

\begin{qcode}
\begin{example}
  Här är ett ensamt exempel.
\end{example}
\end{qcode}

På detta sätt får man en numrering som fortsätter uppåt i hela texten
och möjlighet att referera till exempel utan att man behöver bekymra
sig om att numreringen behöver göras om när man lägger till eller tar
bort exempel.

\label{page:ipa}
För fonetisk skrift med IPA
(% The International Phonetic Association
\textipa{[Di Int@"næS@n@l f@"nEtIk @soUsi"eISn]})
kan man använda paketet \pack{tipa}.

Dokumentation finns lokalt i
% Även /usr/share/doc/tetex-font-tipa-1.3/tipaman.dvi,
% men jag refererar hellre till en pdf.
\path{/local/texmf/doc/tipa/tipaman.pdf}.

Lingvistiska träd ritas enklast med paketet \pack{qtree}.
Använd

\begin{qcode}
\usepackage{etex}
\usepackage{qtree}
\end{qcode}
i dokumentets huvud för att ladda det\footnote{%
  Observera att \pack{etex} måste laddas först när man
  använder \pack{qtree} tillsammans med \class{memoir}.}
och titta i
\path{/local/texmf/tex/latex/qtree/qtreenotes.pdf} för att få en hel
del exempel på hur man sedan ritar träd med det.

\chapter{Hänvisningar och referenser}
\label{chap:refs}

\section{Rekommendationer}

Det finns däremot en stor mängd olika paket och varianter av dessa i
LaTeX för att hantera referenslistan och hänvisningar.
Detta är mina rekommendationer:

\begin{itemize}
\item Använd |\usepackage[round]{natbib}| i dokumenthuvudet.
  Med väljaren |round| får man () istället för~[].
  % Den väljaren behövs inte för svenska, för då är det default.
\item Använd därefter |\bibliographystyle{plainnat}| för att ange stilen på
  referenslistan i en engelskspråkig text. Om du skriver på svenska
  så använd stilen |sweplnat| istället.
\item Lägg in referenslistan i
  % slutet av
  dokumentet med |\bibliography{|\emph{filnamn}|}|
  där filen \emph{filnamn}|.bib| innehåller en BibTeX-databas.
\end{itemize}

I framtiden lär jag rekommendera paketet \pack{biblatex} istället.
Om du har särskilda krav kan det vara värt att titta på det redan nu.

\section{Hänvisningar}

Till \TeX\ finns ett program BibTeX som hanterar bibliografier och
referensförteckningar i ett särskilt format. Sådana filer har namn
som slutar på |.bib| och redigeras i en särskild BibTeX-mode i Emacs.
Där ger man varje uppslag (bok, artikel, \dots) en särskild nyckel
som man sen använder när man skriver hänvisningar i texten.

Se \tex\ den |exempel.bib| som denna text använder.
Där står

\begin{qcode}
@Book{stromquist:upp,
  author =	 {Siv Strömquist},
  title = 	 {Uppsatshandboken},
  publisher = 	 {Hallgren \& Fallgren},
  year = 	 2003,
  address =	 {Uppsala},
  edition =      3
}
\end{qcode}
vilket gör att jag kan referera till \citet{stromquist:upp}
genom att skriva
\begin{qcode}
... kan referera till \citet{stromquist:upp} genom att ...
\end{qcode}

I paketet \pack{natbib} använder man två olika kommandon för
att lägga in referenser: |\citet| och |\citep|.

Detta följer författar-år-modellen för referenser
(Harvardmodellen). Den används oftast inom
lingvistik och rekommenderas också av
\citet[s.~43]{svenskaskrivregler}.
Där består hänvisningarna av författarnamn och år
inne i texten, ibland innanför och ibland utanför parentes,
så det är därför det behövs två olika kommandon.

När man nämner en författare i texten använder man |\citet| för detta,
så
\begin{qcode}
Det senare rekommenderas av \citet{hellmark:th}.
\end{qcode}
ger
\begin{quote}
  Det senare rekommenderas av \citet{hellmark:th}.
\end{quote}
Bara årtalet läggs till inom parentes. 

När hela referensen är ett parentetiskt tillskott används
istället |\citep|.
\begin{qcode}
kapitlet om LaTeX i kurskompendiet till IDS \citep{idslatex}.
\end{qcode}
ger
\begin{quote}
kapitlet om LaTeX i kurskompendiet till IDS \citep{idslatex}.
\end{quote}

Ifall samma författare har skrivit flera verk från samma år läggs det
automatiskt till särskiljande bokstäver.

Eftersom nästan alla använder författar-år-modellen i sina uppsatser
är det den enda som beskrivs här. Det finns dock gott stöd för andra
typer av referenser också.
Eftersom författar-år-modellen är ovanligt lätt att hantera för hand
så funkar det i och för sig rätt bra att skriva ut referenserna för
hand istället, men då får man inte länkar i PDF:en och kan inte få
litteraturförteckningen genererad automatiskt.


\subsection{Hänvisningar till delar av texter}

Ibland vill man referera till en viss del av en text,
\tex\ genom att ange sidnummer.
Det ska man i synnerhet göra när man \emph{citerar} en text
\citep[s.~42]{stromquist:upp}.
Man kan lägga till en sådan anvisning med ett optionellt
argument inom hakparentes till cite-kommandona. Den hänvisning som
finns i detta stycke skrevs
|\citep[s.~42]{stromquist:upp}|.
(Observera tildetecknet för fast mellanslag före sidnumret.)

Härovan används ett komma för att skilja av denna del,
vilket är standardinställningen i \pack{natbib}. Vill
man ha det annorlunda kan man välja den interpunktion
som hänvisningarna använder med kommandot |\bibpunct|.

Här är ett exempel som visar default-värdena.

\begin{qcode}
\bibpunct[, ]{(}{)}{;}{a}{,}{,}
\end{qcode}

Det optionella argumentet (inom hakparenteser) anger skiljetecken
före sidhänvisning. Andra tänkbara värden kan vara |:| eller |:~|.

% För detaljer om de sex argument som kommer därefter, se
% dokumentationen till \class{natbib}. % **referens
% Det av dessa argument där du mest troligt vill ha något annat än
Ett ytterligare argument som du kanske vill ändra på är det
näst sista som anger interpunktion mellan författarnamn och år.
Default är ett komma, men det är också vanligt att inte ha någon
interpunktion alls där.
\Citet[s.~43]{stromquist:upp} förespråkar antingen
\q{(Ekman 1996:27)} eller \q{(Ekman 1996 s.~27)}.
För att få den första formen kan man alltså ändra det
optionella argumentet och det näst sista argumentet:

\begin{qcode}
\bibpunct[:]{(}{)}{;}{a}{}{,}

\citep[27]{ekman}
\end{qcode}

Istället för bara ett kolon förekommer också |:~| i samma position,
för att ge ett (fast) mellanrum efter kolonet.

\section{Litteraturförteckning}

Om man genomgående använder cite-kommandon för att citera
så kan man skapa litteraturförteckningen automatiskt,
så att allt man refererar kommer med i den, men inget annat,
även om det finns fler böcker och artiklar i ens |.bib|-fil.

(Om man skriver ut en referens utan att använda cite-kommandona
av någon anledning bör man därför lägga in |\nocite{|\emph{nyckel}|}|
direkt efter så vet \TeX\ att man gjort en refererens där.)

Exakt vad som skrivs ut i bibliografier i vilken ordning och med vilka
typsnitt varierar mycket mellan olika skolor. Med BibTeX talar man
om vilken stil man vill ha med kommandot
|\bibliographystyle|.
% Jag rekommenderar |plainnat| för uppsatser på engelska.
% (Det är en särskild stil skriven för \pack{natbib} av
% dess författare Patrick~W. Daly.)
% För uppsatser på svenska rekommenderar jag dess svenska
% variant |sweplnat| som är skriven av Lars Engebretsen
% på \textsf{NADA}.

Själva litteraturförteckningen läggs sedan in i slutet av uppsatsen
med kommandot |\bibliography|.

\section{Att köra BibTeX}

När du använder BibTeX så behöver du göra flera körningar för att all
information ska hamna där den behövs. Detta märker Emacs av och
defaultkommandot till |C-c C-c| ändras allteftersom.
När du \TeX{}ar filen får du ett meddelande
\q{You should run BibTeX to get citations right} och mycket riktigt
är \q{BibTeX} defaultkommandot nästa gång du kör. Därefter behöver du
köra om LaTeX, men det vet Emacs också, så fortsätt bara att ge
defaultkommandot tills du får \q{LaTeX: successfully formatted}.

\section{Att editera \texttt{.bib}-filen}

För att lägga till nya uppslag i en |.bib|-fil, använd menyn
\emph{Entry-Types} där. Om du \tex\ väljer \emph{Article in Journal}
får du upp en mall i stil med den i figur~\ref{fig:bibtex}.

\begin{figure}
  \centering
  \begin{qcode}
@Article{,
  author = 	 {},
  title = 	 {},
  journal = 	 {},
  year = 	 {},
  OPTkey = 	 {},
  OPTvolume = 	 {},
  OPTnumber = 	 {},
  OPTpages = 	 {},
  OPTmonth = 	 {},
  OPTnote = 	 {},
  OPTannote = 	 {}
}
\end{qcode}
  \caption{Ett exempel på ett ännu oskrivet BibTeX-uppslag}
  \label{fig:bibtex}
\end{figure}

Fyll i uppslagen och använd |C-j| för att gå
till nästa. Du får ledtext i ekoarean. Fälten som börjar med \q{OPT}
behöver inte fyllas i.

De rekommenderade bibliografistilarna klarar även en del andra fält.
Du kan \tex\ lägga till ett fält \q{url} med en url till en resurs.

När du är klar, så tryck |C-c C-c| för att avsluta. De frivilliga
fält som du inte fyllt i tas då bort. Slutligen får du frågan om
\q{key to use}. Det är denna identifikationsnyckel som du använder
i cite-kommandona. Det är vanligt att man tar med författarnamn,
kanske år och början av titeln, ofta avdelade med kolon, men man kan
välja fritt. I denna text använder jag \tex\ |hellmark:th| för
Christer Hellmarks \emph{\textbf{T}ypografisk \textbf{h}andbok}
och |svenskaskrivregler| för \emph{Svenska skrivregler}.

Om det är flera författarnamn så skilj dem åt med ordet \q{and}.
Det är upp till bibliografistilen hur detta sedan visas i
litteraturförteckningen. (Tanken är nämligen att samma |.bib|-fil ska
kunna användas till många olika slags referenser och
litteraturförteckningar.)

BibTeX kommer att dela upp namnen i förnamn
och efternamn, och kan ibland behöva hjälp med det genom att man
använder klamrar, i stil med
\begin{qcode}
  author = 	 {Anna {Sågvall Hein}},
\end{qcode}
för att inte BibTeX ska tro att Sågvall är ett förnamn. (Det hade
också funkar bra att skriva \q{Sågvall Hein, Anna}.)

Klamrar kan även användas för att skydda versaler i lägen där många
bibliografistilar gör gemener av versaler mitt i titlar.

\chapter{Uppsatsens språk}\label{chap:lang}

\section{Att använda flera språk}

Uppsatsens huvudspråk lär vara svenska eller engelska, men beroende på
ämne kanske du har text på flera olika språk i din uppsats.

För att få den riktigt satt (\tex\ vad gäller avstavningar) så behöver
du tala om för LaTeX vilka språk som används. Lista då alla ingående
språk när du laddar paketet \pack{babel}, \tex\ så här:

\begin{qcode}
\usepackage[english,french,swedish]{babel}
\end{qcode}

Huvudspråket i din uppsats ska listas \emph{sist}.

Sen kan du byta språk i uppsatsen med |\selectlanguage{|\emph{språk}|}|
eller med en omgivning |otherlanguage|.

\subsection{Sammandrag på annat språk}

Ett exempel är om du vill ha sammandrag på ett annat språk, \tex\ ett
engelskt sammandrag i en svensk uppsats:

\begin{qcode}
\selectlanguage{english}
\begin{abstract}
...
\end{abstract}
\selectlanguage{swedish}
...
\end{qcode}

Observera att språkbytet här sker före att |abstract|-omgivningen
börjar så att även dess rubrik ska bli på engelska (\q{Abstract}).

Med användning om omgivningen |otherlanguage| hade detta istället sett
ut så här:

\begin{qcode}
\begin{otherlanguage}{english}
\begin{abstract}
...
\end{abstract}
\end{otherlanguage}
\end{qcode}

\section{Citattecken}

Citattecken skrivs olika på olika språk och det raka
skrivmaskinstecknet |"| används aldrig till citat i satt text.
I satt text finns istället två olika dubbla citattecken, och
på \tex\ engelska används de före respektive efter citat:
``så här''.
Det får man genom att skriva |``| och |''| i
\TeX-filen. På grund av detta är \TeX-moderna i Emacs gjorda så
att när man trycker på |"|-knappen lägger den istället in |``|
eller |''| i bufferten. (I en verbatim-omgivning (se
avsnitt~\ref{sec:verbatim}) kanske du verkligen vill skriva
tecknet~|"|. Tryck då på |"|-tangenten en gång till.)

På andra språk gör man annorlunda. För att få citattecknen rätt,
oavsett på vilket språk man skriver, och även när citat hamnar inuti
andra citat, utan att behöva fundera på det, kan det vara praktiskt
att använda paketet \pack{csquotes} som definierar kommandot
|\enquote| som tar ett argument vilket är texten som ska stå inom
citatet.

Då gör man så här:

\begin{qcode}
\usepackage[babel]{csquotes}	% i dokumenthuvudet
\let\q\enquote

\q{You might just as well say,} added the Dormouse,
who seemed to be talking in his sleep, \q{that
  \q{I breathe when I sleep} is the same thing as
  \q{I sleep when I breathe}!}
\end{qcode}

Väljaren |babel| gör att \pack{csquotes} ger lämpliga citattecken för
det aktuella språket enligt \pack{babel}.
Här definieras ett kortare alias |\q| för |\enquote|. 

Med \pack{babel}-språket |british| skulle detta ge följande:

\begin{quotation}
\begin{otherlanguage}{british}
\q{You might just as well say,} added the Dormouse,
who seemed to be talking in his sleep, \q{that
  \q{I breathe when I sleep} is the same thing as
  \q{I sleep when I breathe}!}\footnote{%
  Citatet är från \emph{Alice i Underlandet}.
  Med |american| skulle istället tvärtom de yttre citattecknen ha
  blivit dubbla och de inre ha varit enkla.}
\end{otherlanguage}
\end{quotation}

På svenska används ibland \q{gåsfötter} som citattecken (som i denna
text). Om man vill använda sådana för svenska får man tala om det för
\pack{csquotes} genom att ge en extra väljare i stil med någon av
dessa två varianter:

\begin{qcode}
% traditionella svenska »gåsfötter» åt samma håll
\usepackage[babel,swedish=guillemets]{csquotes}

% numer allt vanligare »gåsfötter« som pekar inåt
\usepackage[babel,swedish=guillemets*]{csquotes}
\end{qcode}

\section{Språk med andra skriftsystem}

Om du har språkexempel med arabiska eller hebreiska alfabet så använd
paketet \pack{arabtex}. Sök på nätet för info om detta!
För andra skrifter finns det andra lösningar, men kanske inte sådana
som finns installerade här. Berätta för Per vad du behöver!
(Fonetisk skrift nämns på s.\,\pageref{page:ipa}.)

\chapter{Diverse}

\section{Programkod och liknande}
\label{sec:verbatim}

Omgivningen |verbatim| för programkod och liknande som ska återges
precis som den är fungerar lite annorlunda än i standardklasserna.

En skillnad är att TAB-tecken fungerar som åtta steg framåt.
Det normala är annars att en serie TAB-tecken behandlas som ett enda
mellanrum. En annan skillnad är att \textsc{Ascii}-tecknen |`| och~|'|
verkligen visas på detta sätt i |verbatim| och inte som snyggare
citattecken \texttt{`} och~\texttt{'}.

Övriga finesser kan man läsa om i kapitel~16 i Memoir-dokumentationen,
\tex\ om hur man ändrar vilket typsnitt som används för verbatim text
och hur man lägger till radnummer.

\section{Urlar}

Man vill helst inte radbryta urlar, men om man blir tvungen att göra
det ändå vill man inte lägga in några bindestreck vid avstavningen,
eftersom det kan bli oklart om bindestrecket ingår i urlen i så fall.
Använd
\begin{qcode}
\usepackage{url}
...
På \url{http://stp.lingfil.uu.se/datorer/tex/} finns ...
\end{qcode}

Det sköter radbrytning på ett annat sätt. Dessutom kan man skriva
|~| direkt i argumentet för att få med ett sådant tecken i urlen.
(Annars betyder det ju hårt mellanslag i \TeX.)

Urlar skrivs normalt med skrivmaskinsstil. Med
|\urlstyle{same}| kan du ställa om så att de istället skrivs med samma
stil som den omgivande texten.

Samma kommando kan vara lämpligt även för att ange filnamn,
dator\-post\-adresser och liknande, fast hellre varianten |\path|
för en sökväg (som inte gör en webblänk från pdf:en).

\section{Matematik}
Matematik är vad \TeX\ är allra bäst på, och \class{stpthesis}
tillför ingenting särskilt vad gäller detta, så information om sådant
får den som behöver det söka på annat håll, förutom ett par
anmärkningar om ekvationsnummer, eftersom \class{stpthesis} använder
samma numrering för matematiska ekvationer och lingvistiska exempel.

Normalt numrerar LaTeX ekvationer kapitel för kapitel och ger dem
nummer som~5.1. Eftersom språkvetare brukar numrera exempel i en enda
följd genom hela texten ändras detta av \class{stpthesis}. Om du vill
numrera kapitel för kapitel så använd väljaren |eqnobychapter|.

Normalt sitter ekvationsnummer till höger. Exempelnummer sitter dock
alltid till vänster. Om du har en uppsats med både ekvationer och
exempel kan du överväga att använda väljaren |leqno| för att få även
ekvationsnumren till vänster för att få ett mer enhetligt utseende.

\section{Typsnitt}

Typsnitten är normalt de som ingår i Uppsala universitets grafiska
profil, nämligen Berling antikva och Gill Sans. Universitet
har licens för att använda dessa för universitets verksamhet, vilket
även gäller studenter som gör saker \q{inom universitets verksamhet}.

Genom att ange en väljare direkt till klassen kan man ange vilka
fonter man vill använda. Defaultet är |uufonts| och de andra
möjligheterna är |times| och |cm|.
Med |times| används standardpostscriptfonterna
times/helvetica/courier, medan |cm| står för \emph{computer
modern} vilket är den typsnittsfamilj som annars är standard i \TeX.

\section{Memoir är flexibelt}

Om du inte tycker om utseendet så går det att ändra på rätt mycket.
Klassen \class{memoir} som \class{stpthesis} bygger på är mycket
mer flexibel än standardklasserna.
Se \path{/usr/share/texmf/doc/latex/memoir/memman.pdf}
för dokumentationen till \class{memoir} för närmare detaljer. (Det är
en bok på över 250~sidor och finns utskriven i en tidskriftsamlare med
etiketten \TeX\ i stora datorsalen.)

\chapter{Problem}

\section{Placering av figurer och tabeller}\label{sec:placering}

Vid inläggning av figurer och tabeller kan man ange en
\emph{float position}. Om man inte anger detta får man en
defaultplacering som vanligen blir högst upp eller längst ner på en
sida där figuren ryms, helst den då aktuella sidan.

Algoritmen som sköter denna placering är komplicerad. Den tar hänsyn
bl.\,a. till att figurerna ska komma i samma ordning som de ligger i
texten, att alla sidor ska vara lagom välfyllda och inte ha för stor
del figurer på sig.

I allmänhet är ett starkt råd att om du inte blir nöjd med var LaTeX
placerar en viss figur eller tabell, så vänta med att krångla med
detta det tills din text är i princip slutgiltig, i alla fall för det
kapitel det gäller! Annars kan det ändå visa sig att en smärre
ändring i texten som gör ett avsnitt lite längre eller kortare kommer
att leda till helt nya förutsättningar för sidbrytningen, och då har
du bara slösat bort tid på att stångas mot hur LaTeX gör detta.

\section{Dåliga raddelningar}
Ibland är det svårt att dela av raderna på bra ställen, \tex\ om du
har långa ord som \TeX\ inte vet hur den ska avstava. Normalt är \TeX\
mycket nogräknad med hur snygga raderna måste bli, och om den inte
lyckas uppfylla sina höga krav  ger den upp och låter en bit sticka ut
i marginalen för att användaren ska fixa till det.

I \class{stpthesis} är det ändrat så att \TeX\ ändå gör så gott det
går.  I nödfall blir mellanrummen mellan ord då utsträckta alldeles för
mycket.
Om du ger dig på att förbättra sådana dåliga raddelningar blir det
enklare att se vilka rader \TeX\ har problem med om du gör den nogräknad
igen. Det kan du göra med kommandot |\fussy|.
% Om du använder väljaren |draft| till \class{stpthesis}
% för att visa att din text än så länge
% bara är ett utkast så är \TeX\ också nogräknad med orddelningar och
% visar dessutom tydligt var det är problem med en svart blaffa i
% marginalen.

Ibland ger avstavningsreglerna felaktiga resultat. Då kan du tala om
uttryckligen hur ord ska avstavas med ett kommando |\hyphenation| före
texten:

\begin{qcode}
\hyphenation{sats-yta sats-ytan}
\end{qcode}

\bibliography{exempel}


\end{document}
